%% LyX 2.3.0 created this file.  For more info, see http://www.lyx.org/.
%% Do not edit unless you really know what you are doing.
\documentclass[11pt]{article}
\renewcommand{\rmdefault}{cmr}
\renewcommand{\sfdefault}{cmss}
\usepackage[latin9]{inputenc}
\usepackage{geometry}
\geometry{verbose}
\setlength{\parskip}{6pt}
\setlength{\parindent}{0pt}
\usepackage[authoryear]{natbib}
\usepackage[unicode=true,
 bookmarks=false,
 breaklinks=false,pdfborder={0 0 1},backref=section,colorlinks=false]
 {hyperref}

\makeatletter

%%%%%%%%%%%%%%%%%%%%%%%%%%%%%% LyX specific LaTeX commands.
%% Because html converters don't know tabularnewline
\providecommand{\tabularnewline}{\\}

%%%%%%%%%%%%%%%%%%%%%%%%%%%%%% User specified LaTeX commands.


%%%%%%%%%%%%%%%%%%%%%%%%%%%%%%%%%%%%%%%%%%%%%%%%%%%%%%%%%%%%%%%%%%%%%%%%%%%%%%%%%%%%%%%%%%%%%%%%%%%%%%%%%%%%%%%%%%%%%%%%%%%%%%%%%%%%%%%%%%%%%%%%%%%%%%%%%%%%%%%%%%%%%%%%%%%%%%%%%%%%%%%%%%%%%%%%%%%%%%%%%%%%%%%%%%%%%%%%%%%%%%%%%%%%%%%%%%%%%%%%%%%%%%%%%%%%
\usepackage{graphicx}
\usepackage{float}
\usepackage{url}
\usepackage{array}
\newcolumntype{L}[1]{>{\raggedright\let\newline\\\arraybackslash\hspace{0pt}}m{#1}}

\title{\Huge Econometrics I}
\author{\Large Chris Conlon}
\date{\Large Fall 2025}

\makeatother

\begin{document}
\maketitle
\begin{center}
\begin{tabular*}{0.9\textwidth}{@{\extracolsep{\fill}}@{\extracolsep{\fill}}lr}
Lecture: Thursday 2:00-5:00pm in KMC 7-191  & \tabularnewline
Office hours: KMC 7-76, by appointment & \tabularnewline
\url{https://calendly.com/ctc5-stern/30min} &  \tabularnewline
Email: \href{mailto:cconlon@stern.nyu.edu}{cconlon@stern.nyu.edu}  &  \tabularnewline

\end{tabular*}
\par\end{center}

\section*{Course Description}

This is an econometrics course for first-year PhD students
 who are interested in doing quantitative research in the
social sciences. The aim of the course is to teach you to use popular
applied econometric methods while developing your theoretical understanding
of those methods.

Topics include least squares, asymptotic theory,
hypothesis testing, instrumental variables, difference-in-differences,
regression discontinuity, treatment effects, panel data, maximum likelihood,
discrete choice models, and structural estimation. Professor 
Chris Conlon will teach Econometrics II in the spring.

\section*{Prerequisites}

I will assume you are proficient in calculus, basic probability and statistics, and linear algebra.

I also assume proficiency in mathematical programming, which you will need to complete the assignments.
You may use any software you like for the assignments.
I recommend \texttt{R} or Python (both open source)  for those who aren't already committed to something else. 
The course will not involve programming instruction, but I will give some examples of working code in \texttt{R}.\\

All of the assignments can (should?) be done using either \texttt{fixest} in \texttt{R} or \texttt{pyfixest} in Python.

If you aren't already familiar, now is a good time to learn to compose work in \texttt{LaTeX}. 
It's also worth looking into applications that faciliate both composing and presenting code such as RMarkdown and Jupyter.


\section*{Materials}

You should own at least one econometrics textbook, and the suggested textbook for this course is
Hansen's \href{https://www.ssc.wisc.edu/~bhansen/econometrics/}{Econometrics}. 
It used to be free as a PDF but now is for sale on Amazon and other bookstores.


Suggested textbook: The calendar below lists chapters from Hansen corresponding to the lecture material. 

Other useful textbooks:
\begin{itemize}
\item \emph{Econometric Analysis} (8th edition) by William Greene. 
\item Angrist and Pischke, \emph{Mostly Harmless Econometrics}: great as a complement to Greene or Hansen, but I don't recommend 
	    it as your primary econometrics reference. 
\item Wooldridge, \emph{Econometric Analysis of Cross Section and Panel Data}: somewhat more advanced, a very useful reference.
\end{itemize}

Other references:
\begin{itemize}
	\item Sargent and Stachurski's Quantitative Economics with Python (\url{https://python.quantecon.org/intro.html})
\end{itemize} 

\section*{Assignments and Grading}

Your grade will be based on four individual assignments (10\% each), two quizzes (10\% each),
and a final research project (40\%).

Assignments will have a mixture of theoretical questions and data-based
questions. For each assignment, you should turn in (1) a PDF document
presenting your results (and showing your work), and (2) your code. You are encouraged to discuss
assignments with your classmates and work together (learning from your peers is very important during your PhD),
but you must turn in your own work (including your own code). Assignments
should be submitted through Brightspace.

Late assignments will receive partial credit with penalties in proportion to how late they are. 
Assignments more than one week late will not receive credit. 

The group project will consist of three deliverables: 
\begin{enumerate}
\item A short description of your proposed project (up to three pages),
submitted in the middle of the semester. 
\item A presentation of your results during the final class session. 
\item A research paper, turned in at the end of the semester (along with
your code).
\end{enumerate}
Your team for the final project should consist of 1-3 students. The
project can be on any topic (subject to my approval). 


For the final project, I suggest choosing
a published paper to reproduce, and in addition to reproducing the
results, find at least one new way to test, extend, or improve on
the paper's econometric analysis.
I encourage you to look beyond incredibly famous papers
with many thousands of citations for two reasons. First, there is usually a large
subsequent literature -- too large for you to become an expert on just for this project.
Second, these term projects can eventually become (or lead to things that will become)
parts of your dissertation or published papers. That's more likely to happen if you
take on a topic that hasn't already been over-studied. 


Any regrade requests should be submitted in writing.

\section*{Getting Help}

My office hours are by appointment -- please do not hesitate to contact me to set up a meeting. 

Also feel free to send me smaller questions by email, and I will try to respond within 48 hours. 


\newpage
\section*{Tentative Outline}
\begin{center}
\begin{tabular}{L{.08\textwidth} L{.08\textwidth}  L{.12\textwidth} L{.72\textwidth}}
Session  & Date  & Chapters & Topics and deliverables \\
 & & (Hansen) & \\
\hline 
\hline 
1  & 9/4 & 2,6 & Probability and Statistics \\
 &  &  & \\
\hline
2  & 9/11  &  3-4 &  Linear Regression  \\
 &  &  & \\
\hline 
3  & 9/18  & 3-4,7 & Linear Regression  \\
 &  &  & \\
\hline 
4  & 9/25  & 6,7,9  &  Inference, Standard Errors, Testing  \\
 &  &  & \textbf{\textit{Assignment 1 due}}  \\
\hline 
5 & 10/2 & 18 & (Quasi-)Experiments, Endogeneity, Treatment Effects\\
 &  &  &  \\
\hline 
6  & 10/9 &  12  & Instrumental Variables, Simultaneity \\
 &  &   &  \textbf{\textit{Assignment 2 due}}  \\
\hline 
7 & 10/16  & 12 & Instrumental Variables, Simultaneity  \\
 &  &    & \textbf{\textit{In-class quiz}}  \\
\hline 
8  & 10/23 & 17  &  Panel Data, Fixed and Random Effects \\
 &  &  & \\
\hline 
9  & 10/30  &  5 & Maximum Likelihood, Heckman Selection Correction  \\
 &  & &  \textbf{\textit{Final project proposals due}}  \\
\hline 
10  & 11/6  & 22  &  Binary and Discrete Choice \\
 &  &  &  \textbf{\textit{Assignment 3 due}}   \\
\hline 
 11 & 11/13  & 22, 23 & Count Data, Model Selection and Machine Learning  \\
 &  &  & \\
\hline 
12  & 11/20  &  & Introduction to Structural Estimation  \\
 &  &  & \textbf{\textit{In-class quiz}}  \\
 &  &  &  \textbf{\textit{Assignment 4 due}}   \\
\hline 
  & 11/27  &   & No class - Thanksgiving \\
 &  &  &   \\
\hline 
 13 & 12/4 & & \textbf{\textit{Final project presentations}} \\
  &  &  & \\
 \hline
   & 12/11 & &  \textbf{\textit{Final projects due}}\\
 &  &  & \\
\hline 
\end{tabular}
\par\end{center}
\end{document}
