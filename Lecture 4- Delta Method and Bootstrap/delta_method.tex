\documentclass[aspectratio=169]{beamer}
\usetheme{metropolis}

%\usetheme{Darmstadt}
%\usepackage{times}
%\usefonttheme{structurebold}

\usepackage[english]{babel}
%\usepackage[table]{xcolor}
\usepackage{amsmath}
\usepackage{pgf,pgfarrows,pgfnodes,pgfautomata,pgfheaps}
\usepackage{amsmath,amssymb,setspace}
\usepackage[latin1]{inputenc}
\usepackage[T1]{fontenc}
\usepackage{relsize}

%\usepackage{pxfonts}

\usepackage[absolute,overlay]{textpos} 
\newenvironment{reference}[2]{% 
  \begin{textblock*}{\textwidth}(#1,#2) 
      \footnotesize\it\bgroup\color{red!50!black}}{\egroup\end{textblock*}} 

\DeclareMathSizes{10}{10}{6}{6} 


\title [Bootstrap]{Delta Method}
\author{C.Conlon}
\institute{Applied Econometrics II}
\date{\today}
\setbeamerfont{equation}{size=\tiny}
\begin{document}

\begin{frame}
\titlepage
\end{frame}


\begin{frame}{Bootstrap and Delta Method}
\begin{itemize}
\item We know how to construct confidence intervals for parameter estimates:  $\hat{\theta}_k \pm 1.96 SE(\hat{\theta}_k)$
\item Often we are asked to construct standard errors or confidence intervals around model outputs that are not just parameter estimates: ie:  $g(x_i,\hat{\theta})$.
\item Sometimes we can't even write $g(x_i,\theta)$ as an explicit function of $\theta$ ie: $\Psi(g(x_i,\theta),\theta) = 0$.
\item Two options:
\begin{enumerate}
\item Delta Method
\item Bootstrap
\end{enumerate}
\end{itemize}
\end{frame}

\begin{frame}{Delta Method}
Delta method works by considering a \alert{Taylor Expansion} of $g(x_i,\theta)$.
\begin{eqnarray*}
g(z) \approx g(z_0) + g'(z_0)(z-z_0) + o(||z-z_0||)
\end{eqnarray*}
Assume that $\theta_n$ is asymptotically normally distributed so that:
\begin{eqnarray*}
\sqrt{n} (\theta_n - \theta_0) \sim N(0,\Sigma)
\end{eqnarray*}
(How do we get this: OLS? GMM? MLE?).\\

Then we have that 
\begin{eqnarray*}
\sqrt{n} (g(\theta_n) - g(\theta_0)) \sim N(0,D(\theta)' \Sigma  D(\theta))
\end{eqnarray*}
Where $D(\theta) = \frac{\partial g(x_i, \theta)}{\partial \theta}$ is the Jacobian of $g$ with respect to theta evaluated at $\theta$.\\
We need $g$ to be continuously differentiable around the center of our expansion $\theta$.
\end{frame}


\begin{frame}{Delta Method: Examples}
Start with something simple: $g(\theta)= \overline{X}_1\cdot \overline{X}_2$ with $(X_{1i},X_{2i}) \sim IID$.
We know the CLT applies so that:
\begin{eqnarray*}
\sqrt{n}
\begin{pmatrix}
\overline{X}_1 - \mu_1\\
\overline{X}_2 - \mu_2
\end{pmatrix} &\sim  N
\begin{bmatrix}
\begin{pmatrix}
0\\
0
\end{pmatrix},
\Sigma
\end{bmatrix}
\end{eqnarray*}
The Jacobian is just $D(\theta) =  \begin{pmatrix}\frac{\partial g(\theta)}{\partial \theta_1} \\ \frac{\partial g(\theta)}{\partial \theta_2}  \end{pmatrix} =  \begin{pmatrix}s_2\\ s_1 \end{pmatrix}$\\
So,
\begin{eqnarray*}
V(Y) = D(\theta)' \Sigma D(\theta) =\begin{pmatrix} \mu_2 & \mu_1 \end{pmatrix}  \begin{pmatrix} \sigma_{11} & \sigma_{12} \\ \sigma_{21} & \sigma_{22} \end{pmatrix} \begin{pmatrix} \mu_2 \\\mu_1  
\end{pmatrix} \\
\sqrt{n} ( \overline{X}_1 \overline{X}_2 - \mu_1 \mu_2) \sim N(0,\mu_2^2 \sigma_{11}^2 + 2 \mu_1 \mu_2 \sigma_{12}  + \mu_1^2 \sigma_{22}^2)
\end{eqnarray*}
\end{frame}

\begin{frame}{Delta Method: Examples}
Think about a simple logit:
\begin{eqnarray*}
P(Y_i=1 | X_i ) = \frac{\exp^{\beta_0 + \beta_1 X_i}}{1+\exp^{\beta_0 + \beta_1 X_i}}  \quad P(Y_i=0 | X_i ) = \frac{1}{1+\exp^{\beta_0 + \beta_1 X_i}} 
\end{eqnarray*}
Remember the ``trick'' to use GLM (log-odds):
\begin{eqnarray*}
\log P(Y_i=1 | X_i) - \log P(Y_i=0 | X_i) = \beta_0 + \beta_1 X_i
\end{eqnarray*}
\begin{itemize}
\item Suppose that we have estimated $\hat{\beta_0},\hat{\beta_1}$ via GLM/MLE but we want to know the confidence interval for the probability: $P(Y_i=1 | X_i,\hat{\theta})$
\item The derivatives are a little bit tricky, but the idea is the same.
\item This is what STATA should be doing when you type: \tt{mfx, compute}
\end{itemize}
\end{frame}

\begin{frame}{Delta Method: Other Examples}
Often we have a regression like:
\begin{eqnarray*}
log Y_i = \beta_0 + \beta_1 X_i + \gamma Income_i + \epsilon_i
\end{eqnarray*}
And we are interested in $\beta_1 / \gamma$ so that we have $\beta_i$ in units of ``dollars''. Again Delta Method Works fine here.\\
\end{frame}


\begin{frame}{Delta Method: Some Failures}
But we need to be careful.  Suppose that $\theta \approx 0$ and 
\begin{itemize}
\item $g(x)  = |X|$
\item $g(x)  = 1/X$
\item $g(x)  = \sqrt{X}$
\end{itemize}
These situations can arise in practice when we have weak instruments or other problems.
\end{frame}



\end{document}