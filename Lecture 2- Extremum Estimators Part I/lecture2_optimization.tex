\documentclass[handout, aspectratio=169, 10pt]{beamer}

% packages
\usepackage{newpxmath} % math font is Palatino compatible
% \usepackage[nomath]{fontspec}

\usepackage{setspace}
\usepackage{xcolor}
\usepackage{soul} % for \st
\usepackage{hyperref} % for links
\definecolor{links}{HTML}{2A1B81}
\hypersetup{colorlinks,linkcolor=,urlcolor=links}


% table stuff
\usepackage{chronosys}
\usepackage{verbatim}
% \pagenumbering{arabic}
\usepackage{tabularx}
\usepackage{booktabs}
\usepackage{ragged2e}
\usepackage{mathtools}

% R Code
\usepackage{listings}
\usepackage{courier}
\lstset{basicstyle=\scriptsize\ttfamily,breaklines=true}
\lstset{framextopmargin=50pt,frame=bottomline}

% themes
\usetheme[progressbar=frametitle, block=fill]{metropolis}
\useoutertheme{metropolis}
\useinnertheme{metropolis}

% colors
\definecolor{dimwhite}{rgb}{0.99, 0.99, 0.99}
\definecolor{charcoal}{rgb}{0.21, 0.27, 0.31}
\definecolor{slategray}{rgb}{0.44, 0.5, 0.56}
\definecolor{dimgray}{rgb}{0.41, 0.41, 0.41}
\definecolor{bleudefrance}{rgb}{0.19, 0.55, 0.91}

% beamer options
\setbeamercolor{author}{fg=charcoal}
\setbeamercolor{background canvas}{bg=white}
\setbeamercolor{section in toc}{fg=charcoal}
\setbeamercolor{subsection in toc}{fg=dimgray}
\setbeamercolor{frametitle}{bg=dimwhite, fg=charcoal}
\setbeamercolor{progress bar}{fg=slategray, bg=fg!50!black!30}
\setbeamercovered{transparent}
\setbeamertemplate{itemize items}[triangle]
\setbeamertemplate{itemize subitem}[circle]
\setbeamertemplate{itemize subsubitem}[square]
\setbeamersize{text margin left=7mm,text margin right=7mm} 

% new commands
\newcommand{\q}[1]{``#1''}
\newcommand{\hs}[1]{\textsc{\hfill\scriptsize\color{dimgray}#1}}
\newcommand{\g}[1]{{\color{gray}#1}}
\newcommand{\dg}[1]{{\color{dimgray}#1}}
\newcommand{\sg}[1]{{\color{slategray}#1}}
\newcommand{\bdf}[1]{{\color{bleudefrance}#1}}
\newcommand{\itemcolor}[1]{\renewcommand{\makelabel}[1]{\color{#1}\hfil ##1}}
\newcommand\Wider[2][2em]{
\makebox[\linewidth][c]{
  \begin{minipage}{\dimexpr\textwidth+#1\relax}
  \raggedright#2
  \end{minipage}
  }
}

% misc
\linespread{1.35}

% minted (risky)
\usepackage[cache=false]{minted}

% Math stuff
\newcommand{\norm}[1]{\left\lVert#1\right\rVert}
\newcommand{\R}{\mathbb{R}}
\newcommand{\E}{\mathbb{E}}
\newcommand{\V}{\mathbb{V}}
\newcommand{\probP}{\text{I\kern-0.15em P}}
\newcommand{\ol}{\overline}
%\newcommand{\ul}{\underline}
\newcommand{\pp}{{\prime \prime}}
\newcommand{\ppp}{{\prime \prime \prime}}
\newcommand{\policy}{\gamma}
\newcommand{\plim}{ \overset{p}{\to}}
\newcommand{\hnot}{ \overset{H_0}{\to}}

% Causal Graphs
\usetikzlibrary{shapes,decorations,arrows,calc,arrows.meta,fit,positioning}
\tikzset{
    -Latex,auto,node distance =1 cm and 1 cm,semithick,
    state/.style ={ellipse, draw, minimum width = 0.7 cm},
    point/.style = {circle, draw, inner sep=0.04cm,fill,node contents={}},
    bidirected/.style={Latex-Latex,dashed},
    el/.style = {inner sep=2pt, align=left, sloped}
}

% \usepackage{slashbox}
\title{Lecture 2: Maximum Likelihood and Friends}
\author{Chris Conlon }
\institute{NYU Stern }


\date{\today}

\begin{document}
\maketitle


\section*{Computing Maximum Likelihood Estimators}

\begin{frame}{Newton's Method for Root Finding}
Consider the Taylor series for $f(x)$ approximated around $f(x_0)$:
\begin{align*}
f(x) \approx f(x_0) + f'(x_0) \cdot (x-x_0) + f''(x_0) \cdot (x-x_0)^2 + o_p(3)
\end{align*}
Suppose we wanted to find a \alert{root} of the equation where $f(x^{*})=0$ and solve for $x$:
\begin{align*}
0 &= f(x_0) + f'(x_0) \cdot (x-x_0) \\
x_1 &= x_0-\frac{f(x_0)}{f'(x_0)} 
\end{align*}
This gives us an \alert{iterative} scheme to find $x^{*}$:
\begin{enumerate}
\item Start with some $x_k$. Calculate $f(x_k),f'(x_k)$
\item Update using $x_{k+1} = x_k - \frac{f(x_k)}{f'(x_k)} $
\item Stop when $|x_{k+1}-x_{k}| < \epsilon_{tol}$.
\end{enumerate}
\end{frame}

\begin{frame}{Newton-Raphson for Minimization}
We can re-write \alert{optimization} as \alert{root finding};
\begin{itemize}
\item We want to know $\hat{\theta} = \arg \max_{\theta} \ell(\theta)$.
\item Construct the FOCs $\frac{\partial \ell}{\partial \theta}=0 \rightarrow$  and find the zeros.
\item How? using Newton's method! Set $f(\theta) = \frac{\partial \ell}{\partial \theta}$
\end{itemize}
\begin{align*}
\theta_{k+1} &= \theta_k -  \left[ \frac{\partial^2 \ell}{\partial \theta^2}(\theta_k) \right]^{-1} \cdot \frac{\partial \ell}{\partial \theta}(\theta_k)
\end{align*}
The SOC is that $ \frac{\partial^2 \ell}{\partial \theta^2} >0$. Ideally at all $\theta_k$.\\
This is all for a \alert{single variable} but the \alert{multivariate} version is basically the same.
\end{frame}


\begin{frame}{Newton's Method: Multivariate}
Start with the objective $Q(\theta) = - \ell(\theta)$:
\begin{itemize}
\item Approximate $Q(\theta)$ around some initial guess $\theta_0$ with a quadratic function
\item Minimize the quadratic function (because that is easy) call that $\theta_1$
\item Update the approximation and repeat.
\begin{align*}
\theta_{k+1} = \theta_k - \left[ \frac{\partial^2 Q}{\partial \theta \partial \theta'} \right]^{-1}\frac{\partial Q}{\partial \theta}(\theta_k)
\end{align*}
\item The equivalent SOC is that the {Hessian Matrix} is \alert{positive semi-definite}  (ideally at all $\theta$).
\item In that case the problem is \alert{globally convex} and has a \alert{unique maximum} that is easy to find.
\end{itemize}
\end{frame}


\begin{frame}{Newton's Method}
We can generalize to Quasi-Newton methods:
\begin{align*}
\theta_{k+1} = \theta_k -  \lambda_k \underbrace{\left[ \frac{\partial^2 Q}{\partial \theta \partial \theta'} \right]^{-1}}_{A_k} \frac{\partial Q}{\partial \theta}(\theta_k)
\end{align*}
Two Choices:
\begin{itemize}
\item Step length $\lambda_k$
\item Step direction $d_k=A_k \frac{\partial Q}{\partial \theta}(\theta_k)$
\item Often rescale the direction to be unit length $\frac{d_k}{\norm{d_k}}$.
\item If we use $A_k$ as the true Hessian and $\lambda_k=1$ this is a \alert{full Newton step}.
\end{itemize}
\end{frame}

\begin{frame}{Newton's Method: Alternatives}
Choices for $A_k$
\begin{itemize}
\item $A_k= I_{k}$ (Identity) is known as \alert{gradient descent} or \alert{steepest descent}
\item BHHH. Specific to MLE. Exploits the \alert{Fisher Information}.
\begin{align*}
A _ { k } 
&= \left[ \frac { 1 } { N } \sum _ { i = 1 } ^ { N } \frac { \partial \ln f } { \partial \theta } \left( \theta _ { k } \right) \frac { \partial \ln f } { \partial \theta ^ { \prime } } \left( \theta _ { k } \right) \right] ^ { - 1 }\\
&=- \mathbb { E } \left[ \frac { \partial ^ { 2 } \ln f } { \partial \theta \partial \theta ^ { \prime } } \left( Z , \theta ^ { * } \right) \right] 
= \mathbb { E } \left[ \frac { \partial \ln f } { \partial \theta } \left( Z , \theta ^ { * } \right) \frac { \partial \ln f } { \partial \theta ^ { \prime } } \left( Z , \theta ^ { * } \right) \right]
\end{align*}
\item Alternatives \alert{SR1} and \alert{DFP} rely on an initial estimate of the Hessian matrix and then approximate an update to $A_k$.
\item Usually updating the Hessian is the costly step.
\item Non invertible Hessians are bad news.
\end{itemize}
\end{frame}

\section{EM Algorithm and Mixtures}

\begin{frame}
\frametitle{Estimating Finite Mixtures}
\begin{itemize}
\item In practice estimating finite mixture models can be tricky.
\item A simple example is the mixture of normals (incomplete data likelihood)
\begin{eqnarray*}
f(x_1,\ldots,x_n | \theta) = \prod_{i=1}^N \sum_{k=1}^K \pi_k f(x_i | \mu_k, \sigma_k)
\end{eqnarray*}
\item We need to find both mixture weights $\pi_k = Pr(z_k)$ and the components $(\mu_k,\sigma_k)$ the weights define a valid probabiltiy measure $\sum_k \pi_k = 1$.
\item Easy problem is \alert{label switching}. Usually it helps to order the components by say decreasing $\pi_1 > \pi_2 > \ldots$ or  $\mu_1 > \mu_2 > \ldots$ 
\item The real problem is that which component you belong to is unobserved. We can add an extra indicator variable $z_{ik} \in \{0,1\}$.
\item We don't care about $z_{ik}$ per-se so they are \alert{nuisance parameters}.
\end{itemize}
\end{frame}

\begin{frame}
\frametitle{Estimating Finite Mixtures}
\begin{itemize}
\item We can write the complete data log-likelihood (as if we observed $z_{ik}$):
\begin{eqnarray*}
\ell(x_1,\ldots,x_n | \theta) = \sum_{i=1}^N  \log \left( \sum_{k=1}^K I[z_i = k]  \pi_k f(x_i , \mu_k, \sigma_k) \right)
\end{eqnarray*}
\item We can instead maximized the expected log-likelihood where we take the expectation $E_{z|\theta}$
\begin{eqnarray*}
\alpha_{ik}(\theta) = Pr(z_{ik} =1 | x_i,\theta) = \frac{f_k(x_i,z_k,\mu_k,\sigma_k) \pi_k }{\sum_{m=1}^K f_m(x_i,z_m,\mu_m,\sigma_m) \pi_m}
\end{eqnarray*}
\item Now we have a probability $\hat{\alpha}_{ik}$ that gives us the probability that $i$ came from component $k$. We also compute $\hat{\pi}_k = \frac{1}{N} \sum_{i=1}^N \alpha_{ik}$
\end{itemize}
\end{frame}

\begin{frame}
\frametitle{EM Algorithm}
\begin{itemize}
\item Treat the $\hat{\alpha}_k(\theta^{(q)})$ as data and maximize to find $\mu_k,\sigma_k$ for each $k$
\begin{eqnarray*}
\hat{\theta}^{(q+1)} = \arg \max_{\theta}  \sum_{i=1}^N  \log \left( \sum_{k=1}^K \hat{\alpha}_k(\theta^{(q)}) f(x_i | z_{ik}, \theta ) \right)
\end{eqnarray*}
\item We iterate between updating $\hat{\alpha}_k(\theta^{(q)})$ (E-step) and $\hat{\theta}^{(q+1)}$ (M-step)
\item For the mixture of normals we can compute the M-step very easily:
\begin{eqnarray*}
\mu_k^{(q+1)} &=& \frac{1}{N} \sum_{i=1}^N \hat{\alpha}_k(\theta^{(q)}) x_{i}\\
\sigma_k^{(q+1)} &=& \frac{1}{N} \sum_{i=1}^N \hat{\alpha}_k(\theta^{(q)}) (x_{i} - \overline{x})^2 \\
\end{eqnarray*}
\end{itemize}
\end{frame}

\begin{frame}
\frametitle{EM Algorithm}
\begin{itemize}
\item EM algorithm has the advantage that it avoids complicated integrals in computing the expected log-likelihood over the missing data.
\item For a large set of families it is proven to converge to the MLE
\item That convergence is \alert{monotonic} and \alert{linear}. (Newton's method is quadratic)
\item This means it can be slow, but sometimes $\nabla_{\theta} f (\cdot)$ is really complicated.
\end{itemize}
\end{frame}

\section*{Thanks!}

\end{document}